%! Tex program = xelatex   
\documentclass{article}
\usepackage[left=2cm, right=2cm, lines=45, top=0.8in, bottom=0.7in]{geometry}
\usepackage{xeCJK}
\usepackage{amsmath}
\usepackage{booktabs} %表格
\usepackage{forest}
\setmainfont{Times New Roman}
\setCJKmainfont{Songti SC}
\setCJKfamilyfont{song}{Songti SC}
%-----------------------伪代码------------------
\usepackage{algorithm}  
\usepackage{algorithmicx}  
\usepackage{algpseudocode}  
\floatname{algorithm}{Algorithm}  
\renewcommand{\algorithmicrequire}{\textbf{Input:}}  
\renewcommand{\algorithmicensure}{\textbf{Output:}} 
\usepackage{lipsum}  
\makeatletter
\newenvironment{breakablealgorithm}
  {% \begin{breakablealgorithm}
  \begin{center}
     \refstepcounter{algorithm}% New algorithm
     \hrule height.8pt depth0pt \kern2pt% \@fs@pre for \@fs@ruled
     \renewcommand{\caption}[2][\relax]{% Make a new \caption
      {\raggedright\textbf{\ALG@name~\thealgorithm} ##2\par}%
      \ifx\relax##1\relax % #1 is \relax
         \addcontentsline{loa}{algorithm}{\protect\numberline{\thealgorithm}##2}%
      \else % #1 is not \relax
         \addcontentsline{loa}{algorithm}{\protect\numberline{\thealgorithm}##1}%
      \fi
      \kern2pt\hrule\kern2pt
     }
  }{% \end{breakablealgorithm}
     \kern2pt\hrule\relax% \@fs@post for \@fs@ruled
  \end{center}
  }
\makeatother
%------------------------代码-------------------
\usepackage{xcolor} 
\usepackage{listings} 
\usepackage{fontspec}
\newfontfamily\menlo{Menlo}
\setmonofont[Mapping={}]{Monaco} 
\definecolor{mygreen}{rgb}{0,0.6,0}
\definecolor{mygray}{rgb}{0.5,0.5,0.5}
\definecolor{mymauve}{rgb}{0.58,0,0.82}
\lstset{ %
backgroundcolor=\color{white},   % choose the background color
basicstyle=\footnotesize\ttfamily,        % size of fonts used for the code
columns=fullflexible,
breaklines=true,                 % automatic line breaking only at whitespace
captionpos=b,                    % sets the caption-position to bottom
tabsize=4,
commentstyle=\color{mygreen},    % comment style
escapeinside={\%*}{*)},          % if you want to add LaTeX within your code
keywordstyle=\color{blue},       % keyword style
stringstyle=\color{mymauve}\ttfamily,     % string literal style
frame=single,
rulesepcolor=\color{red!20!green!20!blue!20},
numbers=left,
 numberstyle=\tiny\menlo
% identifierstyle=\color{red},
% language=c++,
}
\begin{document}
\title{区块链第一次书面作业}
\author{朱浩泽 1911530 计算机科学与技术}
\maketitle
\section{多元 Merkle 树问题}
\subsection{问题a}
\large
画出其Merkle树如下图所示:\\
\begin{forest}
[
  Top\ Hash
  [Hash\ 0[T1][T2][T3]]
  [Hash\ 1[T4][T5][T6]]
  [Hash\ 2[T7][T8][T9]]
]
\end{forest}
\\通过上图可以看出,如果Alice想向Bob证明T4 在 S 中,则先需要T5和T6的值,通过Hash函数算出Hash\ 1的值,再利用Hash\ 0和Hash\ 2的值同算出的计算出的Hash\ 1的值通过Hash函数计算出Top\ Hash的值,与获得的可行的Top\ Hash值进行对比,便可检验出T4 在 S 中。其中我们利用了T5、T6、Hash\ 0、Hash\ 1的值。
\subsection{问题b}
\large
记一共有$n$个元素,每个非叶节点最多可以有$k$个子节点,则此Merkle树有$\lceil \log_kn \rceil +1$层,除去Top\ Hash,每一层的证明需要$(k-1)$个元素,所以其证明长度为$(k-1)\times (\lceil \log_kn \rceil +1-1)=(k-1)\times \lceil \log_kn \rceil$。
\subsection{问题c}
\begin{align*}
&\lim_{n\rightarrow \infty}\frac{(3-1)\times \log_3n}{(2-1)\times \log_2n}\\
=&\lim_{n\rightarrow \infty}2\times \frac{\log_3n}{\log_2n}\\
=&\lim_{n\rightarrow \infty}2\times \frac{\log_n2}{\log_n3}\\
=& 2\times \log_32\\
=& 1.2619 > 1
\end{align*}
由此可见,最好使用二叉默克尔树。
\newpage
\section{分层确定性钱包}
\large 
如果将$H(k||i)$换成$k+i$,第$i$个公钥为$g^{x_i} = g^{k+i}g^y$,则第$i+1$个公钥为$g^{x_{i+1}} = g^{k+i+1}g^y = g^{x_i}g$\\
于此同时,第$i$个私钥为$x_i = y + k + i$,则第$i+1$个私钥为$x_{i+1} = y + k +i + 1 = x_{i} + 1$\\
由此可见,知道一个公钥的来源后,通过简单的乘法计算,可以轻松判定这些公钥来自同一个钱包。与此同时,如果某一天有人猜出来了g的值,通过第$i$个公钥和g的值推算出了$x_i$的值,那么如果我们使用的是$H(k||i)$函数的话,也无法通过这个泄漏的私钥匙推算出其他的私钥;但如果使用的是$k+i$函数的话,只需要简单进行+1计算,便可得知所有的公钥和私钥对,从而盗走所有比特币。所以这种算法也可能会影响安全性。
\end{document}
